$$
f(x)=\frac{n_1(x_1-x)^2 +n_2(x_2-x)^2 + ... + n_p (x_p - x)^2}{N}
$$
$1/N$ is constant

let us expand, for example:
$$
n_1(x_1 - x)^2
=n_1((x_1)^2 - 2(x_1)x + x^2)
$$
if we differentiate this expression, we obtain:
$$
n_1(2x - 2x_1)
$$
In general, for any i in the interval $1...p$.
<p>

if we differentiate $n_i(x_i - x)^2$ we would obtain  $2n_i(x - x_i)$

So, 
$$
f'(x) = \frac{2n_1(x-x_1) + 2n_2(x-x_2) + ... + 2n_p(x-x_p)}{N}
$$
$$
f'(x) = \frac{2n_1x-2n_1x_1 + 2n_2x-2n_2x_2 + ... + 2n_px-2n_px_p}{N}
$$
After some rearranging of terms, we can factorize the above expression into
$$
f'(x) = \frac{2x(n_1+n_2+...n_p)}{N} -\frac{2(n_1 x_1+n_2 x_2+...+n_p x_p)}{N}
$$
As we have $n_1 + n_2 + ... + n_p = N$,
$$
f'(x) = \frac{2xN}{N} -\frac{2(n_1 x_1+n_2 x_2+...+n_p x_p)}{N}
$$
$$
f'(x) = 2x - 2\frac{(n_1 x_1+n_2 x_2+...+n_p x_p)}{N}
$$

$$
f'(x) = 2x - 2\overline{x} = 2(x - \overline{x})
$$
with $\overline{x}$ being the weighted average of the terms

<p>
Remark:
x is a variable and $x_1, x_2, ... x_n$ are well-defined values
To understand the difference:
we can calculate f(t):
$$
f(t)=\frac{n_1(x_1-t)^2 +n_2(x_2-t)^2 + ... + n_p (x_p - t)^2}{N}
$$
